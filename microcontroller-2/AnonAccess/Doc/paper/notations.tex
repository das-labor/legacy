\section{Notations and conventions}
\begin{tabular}{ll}
$a \oplus b$ & $a$ xor $b$ \\
$a \wedge b$ & $a$ bit wise and $b$ \\
$a \vee b$ & $a$ bit wise or $b$ \\
$a \parallel b$ & concatenation of the bit strings $a$ and $b$ \\
$a_{(base)}$ & the constant $a$ is given in base $base$ notation, if not specified the base is 10\\
$H(a)$ & is the value of the hash function SHA-256 of message $a$ \\
$HMAC_{key}(a)$ & is the value of the HMAC-SHA256 MAC function of message $a$ and key $key$ \\
$bit$ & a bit is the basic unit of information; it can only have one of two values, \\ 
& which we consider to be 1 and 0 \\
$byte$ & a byte is considered to be a group of eight bits throughout this document \\
Ki, Mi, Gi & prefixes to units, specifying a multiple of $2^{10} = 1,024$, $2^{20} = 1048,576$ and\\ & $2^{30} = 1,073,741,824$; see \cite{IEC60027-2} for reasons \\
K, M, G & prefixes to units, specifying a multiple of $10^3 = 1,000$, $10^6 = 1,000,000$ and\\ & $10^9=1,000,000,000$ \\ 
\end{tabular}