\section{the Shabea-Cipher}
Shabea (SHA based encryption algorithm) is a SHA-256 based Feistel-Cipher, it was designed to securely encryt data where a SHA-256 implementation is available. It was important to have a small (in program space and memory requirement) when a SHA-256 implementation is availabel.

\begin{algorithm}
\caption{Encryption with Shabea}
\label{algShabeaEnc}
\begin{algorithmic}
\REQUIRE $INPUT = L_0 \parallel R_0$ where $L_0$ and $R_0$ are both 16 bytes large
\REQUIRE $4\leq rounds\leq 255$
\REQUIRE $key$ wich the length (in bits) is $keylength$ of any size
\FOR{$i=0$ to $rounds$}
\STATE $L_{i+1} \leftarrow R_i$
\STATE $R_{i+1} \leftarrow L_i \oplus H(key \parallel 0 \parallel i \parallel R_i)$
\ENDFOR
\STATE $OUTPUT = L_{i+1} \parallel R_{i+1}$
\end{algorithmic}
\end{algorithm}

\begin{algorithm}
\caption{Decryption with Shabea}
\label{algShabeaDec}
\begin{algorithmic}
\REQUIRE $INPUT = L_{rounds} \parallel R_{rounds}$ where $L_{rounds}$ and $R_{rounds}$ are both 16 bytes large
\REQUIRE $4\leq rounds\leq 255$
\REQUIRE $key$ wich the length (in bits) is $keylength$ of any size
\FOR{$i=rounds+1$ downto $1$}
\STATE $R_{i-1} \leftarrow L_i$
\STATE $L_{i-1} \leftarrow R_i \oplus H(key \parallel 0 \parallel i \parallel L_i)$
\ENDFOR
\STATE $OUTPUT = L_0 \parallel R_0$
\end{algorithmic}
\end{algorithm}