\section{AnonAccess Konzept}

\begin{frame}
	\frametitle {Neue Problemstellungen}
	Das Problem der Wartbarkeit
	\begin{itemize}
		\item<2-> Nutzer müssen adressierbar sein.
		\item<3-> Lösungsidee: Nutzer müssen nicht die ganze Zeit adressierbar sein.
	\end{itemize}
\end{frame}

\begin{frame}
	\frametitle {Lösung}
	Änderungen an einem Benutzerkonto werden erst dann angewendet wenn sich der Benutzer anmeldet.
	Die Adressierung findet über einen sogenannten Nickname statt.\\
	Dieser Nickname wird jedoch nirgwendwo im Klartext gespeichert.\\
	Auf der Karte wird ein verschlüsselter Fingerprint des Nicknames gespeichert.\\
\end{frame}

\begin{frame}
	\frametitle {Lösung}
	Sollen die Eigenschaften eines Kontos modifiziert werden,\\
	dann wird ein Eintrag in der Flag-Modify-Database erstellt.
	\begin{itemize}
	\item<2-> Fingerprint des Nicknames
	\item<3-> Änderungsanweisungen
	\end{itemize}
\end{frame}

\begin{frame}
	\frametitle {Lösung}
	Nun wird der Ablauf der Authentifizierung erweitert um:
	\begin{itemize}
	\item<2-> Entschlüsseln des Nickname Fingerprints
	\item<3-> Suche in der Datenbank nach Änderungen für dieses Konto
	\end{itemize}
\end{frame}

