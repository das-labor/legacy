\section{Begriffe und Vorraussetzungen}
\begin{frame}
	\frametitle{Ticket}
	\vspace{1cm}
	Einmal verwendbarer Datensatz zur Authentifikation.\\
	Das Ticket wird vom System überprüft und bei erfolgreiche Prüfung (Authentifikation) wird ein neues Ticket ausgegeben
\end{frame}
\begin{frame}
	\frametitle{Hash-Funktion}
	\vspace{1cm}
	Eine Hashfunktion bildet eine beliebig große Datenmenge eindeutig auf eine Datenmenge fester Größe ab (Message-Digest oder Fingerprint).\\
	Eine wichtige Anforderung an eine derartig Funktion ist die sogenannte Kollisionsfreiheit, d.h. obwohl es prinzipiell mehrere Nachrichten gibt die den gleichen Fingerprint haben ist es schwierig zwei derartige Nachrichten zu finden.
\end{frame}
