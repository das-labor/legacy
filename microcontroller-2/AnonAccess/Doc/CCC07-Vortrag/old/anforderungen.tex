\section{Anforderungen \& Beispiele}
\begin{frame}
	\frametitle{Anforderungen}
	\begin{itemize}
		\item<2-> Wartbarkeit
		\item<3-> Sicherheit
		\item<4-> Anonymität
		\item<5-> Kostengünstig (Komplette Anlage ~100\euro{})
		\item<6-> Transparenz
	\end{itemize}
\end{frame}

\begin{frame}
	\frametitle{Wartbarkeit}
	\begin{itemize}
		\item<2-> Hinzufügen von Nutzern
		\item<3-> Löschen von Nutzern
		\item<4-> Sperren von Nutzern (mit und ohne Karte)
		\item<4-> Privilegien verwalten
	\end{itemize}
\end{frame}
\begin{frame}
	\frametitle{Sicherheit}
	\begin{itemize}
		\item<2-> Zugang beschränken auf berechtigte Personen
		\item<3-> Verhindern des Kopierens der Zugangsberechtigung
		\item<4-> Sicherheit sollte vergleichbar sein mit konventionellen Schlüsseln
	\end{itemize}
\end{frame}

\begin{frame}
	\frametitle{Anonymität}
	\begin{itemize}
		\item<2-> Schutz der Persönlichen Daten auch vor starken Angreifern
		\item<3-> Ein Angreifer soll nicht von der Karte auf die Zugehörigkeit zu einem System schließen können
		\item<4-> Ein Angreifer soll nicht von der Analyse des Systems auf den Nutzerkreis schließen dürfen
		\item<5-> Anonymität sollte vergleichbar sein mit konventionellen Schlüsseln
	\end{itemize}
\end{frame}

\begin{frame}
	\frametitle{Kostengünstig}
	\begin{itemize}
		\item<2-> geringe Kosten für die Hardware des Systems
		\item<3-> geringe Kosten je Nutzer
		\item<4-> einfache Herstellung (für den Nachbau geeignet)
	\end{itemize}
\end{frame}
\begin{frame}
	\frametitle{Transparenz}
	\begin{itemize}
		\item<2-> Software ist OpenSource (GPLv3)
		\item<3-> Schaltpläne für die Hardware liegen offen
		\item<4-> Keine ''Security by Obscurity''
		\item<4-> Die Sicherheit hängt von der Geheimhaltung des Schlüssels ab, nicht von der Geheimhaltung des Verfahrens. (Kerkhofs Gesetz)
	\end{itemize}
\end{frame}

