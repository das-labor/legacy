\section{AnonAccess Konzept}

\begin{frame}
	\frametitle{Physikalische Sicherheit}
	Zum Schutz vor Auslesen der Datenbank \& geheimen Schl"ussel.
	\begin{itemize}
		\item<2-> Sicherung gegen "offnen des Geh"auses
		\item<3-> Sichere L"oschung der Daten
	\end{itemize}
\end{frame}

%
% diesen frame (flags) evtl spaeter erst einblenden?
%

\begin{frame}
	\frametitle{Flags}
	\vspace{1cm}

	\textbf{Flags}
	\\
	Meta Informationen "uber den Zustand eines Nutzers.
	\begin{itemize}
		\item \texttt{active}: Benutzeraccount aktiv
		\item \texttt{permanent}: Benutzeraccount darf beliebig oft verwendet werden
		\item \texttt{hnick}: HMAC des Pseudonyms
	\end{itemize}
\end{frame}
\begin{frame}
	\frametitle{Daten sicher Speichern}
	Man nehme:
	\begin{itemize}
		\item<2-> Eine symmetrische Blockcipher
		\item<3-> Eine Hand voll Hashfunktionen
		\item<4-> Einen kryptografisch Sicheren Zufallsgenerator
	\end{itemize}
\end{frame}

\begin{frame}
	\frametitle{Daten auf der Karte}
	Auf jeder Karte befinden sich folgende\\
	Daten verpackt in einer ASN-1 Struktur:
	\begin{itemize}
		\item<2-> FIX
		\item<3-> ME
	\end{itemize}
\end{frame}
