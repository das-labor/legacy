\section{Mögliche (alternative) Implementationen}
\begin{frame}
	\frametitle{Variante 1}
	Daten auf der Karte:
	\begin{itemize}
		\item<2-> User ID
		\item<3-> Ticket
	\end{itemize}
	Daten im System für jeden Nutzer:
	\begin{itemize}
		\item<4-> Ticket (oder Fingerprint des Tickets)
		\item<5-> Berechtigungen (sog. Flags)
	\end{itemize}
\end{frame}

\begin{frame}
	\frametitle{Variante 1 - Anwendung}
	Normale Authentifizierung:
	\begin{enumerate}
	\item<2-> Auslesen der User ID und des Tickets von der Karte
	\item<3-> Auslesen der des Tickets aus der Nutzerdatenbank
	\item<4-> Überprüfen des Tickets
	\item<5-> Generieren eines neuen Tickets
	\item<6-> Schreiben des neuen Tickets auf die Karte
	\end{enumerate}
\end{frame}

\begin{frame}
	\frametitle{Variante 1 - Nachteile}
	\textbf<2->{Nachteil:}
	\begin{itemize}
		\item<2-> Zugriff ist nur Pseudonym
		\item<3-> Pseudonym ist relativ schwer zu merken
	\end{itemize}
\end{frame}

\begin{frame}
	\frametitle{Variante 2}
	Wie Variante 1, mit folgender Änderungen:
	\begin{itemize}
		\item<2-> Es wird zusätzlich zu dem Ticket auch eine neue User ID generiert.
	\end{itemize}
\end{frame}

\begin{frame}
	\frametitle{Variante 2 - Nachteil}
	\textbf<2->{Nachteil:}
	\begin{itemize}
		\item<2-> Nicht wartbar, da Nutzer nicht mehr ''adressierbar'' sind.
	\end{itemize}
\end{frame}

\begin{frame}
	\frametitle {Neue Problemstellungen}
	\begin{itemize}
		\item<2-> Speicherung von lesbaren Informationen auf der
		Karte ist eine schlechte Idee.
		\item<3-> Die Nutzerdatenbank sollte gesch"utzt werden.
	\end{itemize}
\end{frame}

\begin{frame}
	\frametitle{Das Problem der Wartbarkeit}
	Zwei Alternativen:
	\begin{enumerate}
		\item<2-> Nutzer Pseudonyme
		\item<3-> Timeouts
	\end{enumerate}
\end{frame}
