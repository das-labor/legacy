\documentclass{beamer}

\usepackage{wrapfig}
\usepackage{beamerbaseverbatim}
\usepackage{beamerthemesplit}
\usepackage[latin1]{inputenc}
\usepackage[T1]{fontenc}

\title{Grafikprogrammierung mit OpenGL}
\author{S\"oren Heisrath}
\date{\today}

\begin{document}

\frame{\titlepage}

\section[\"Ubersicht]{}
\frame{\tableofcontents}

% =========================== INTRO ===========================
\section{Wichtige Primitiven}
% \subsection{Linux ab Werk}
\frame
{
	\frametitle{Koordinatensystem}
	\begin{itemize}
		\item X horizontal (Links negativ, Rechts positiv)
		\item Y Vertikal (Unten negativ, Oben positiv)
		\item Z Der Blickrichtung entgegen positiv
	\end{itemize}
}

\frame
{
	\frametitle{Zeichenprimitiven}
	\begin{itemize}
		\item Vertexes
		\item Lines
		\item Triangles
		\item Quads
		\item Polygone
	\end{itemize}
}

\section{Texturierung}
\frame
{
	\frametitle{Dimensionen}
	\begin{itemize}
		\item 1D ''Texturen'' (Linien)
		\item 2D Texturen
		\item 3D Texturen - Multi-Layer Texturen
	\end{itemize}
}

\frame
{
	\frametitle{Blending}
	\begin{itemize}
		\item Alpha-Blending ((Teil)transparenz)
		\item Reflektionen
		\item Fl\"ussige "Uberg\"ange
	\end{itemize}
}

\section{Wichtige Funktionen}
\frame
{
	\large{\texttt{glLoadIdentity()}}
	\\
	R\"uckkehr zum Koordinatenursprung
}

\frame
{
	\large{\texttt{glEnable() / glDisable()}}
	\\
	An/Ausschalten von OpenGL Features
}

\frame
{
	\large{\texttt{glBegin() ... glEnd()}}
	\\
	Zeichenbefehle
}

\frame
{
	\large{\texttt{glRotateXX()}}
	\\
	Rotationsfunktionen - z.B.:\\
	\texttt{glRotate4f (90.0f, 0.0f, 1.0f, 0.0f)}
}

\frame
{
	\large{\texttt{glVertexXX()}}
	\\
	Zeichenfunktion f\"ur ein Vertex - z.B.\\
	\texttt{glVertex3f (1.0f, 1.0f, 0.0f)}
}

\frame
{
	\frametitle{Das erste Dreieck...}
	\texttt{glBegin(GL\_TRIANGLES)\\  glVertex3f (0.0f, 1.0f, 0.0f);\\  glVertex3f (-0.5f, 0.0f, 0.0f);\\  glVertex3f (0.5f, 0.0f, 0.0f);\\glEnd()}
}

\frame
{
	\large{\texttt{glTexCoordXX()}}
	z.B. \texttt{glTexCoord2f (0.0f, 1.0f)}
}

\frame
{
	\frametitle{Das erste Dreieck mit Texturen...}
	\texttt{glBegin(GL\_TRIANGLES)\\  glTexCoord2f (0.0f, 0.0f);\\  glVertex3f (0.0f, 1.0f, 0.0f);\\  glTexCoord2f (0.0f, 1.0f);\\  glVertex3f (-0.5f, 0.0f, 0.0f);\\  glTexCoord2f (1.0f, 0.0f);\\  glVertex3f (0.5f, 0.0f, 0.0f);\\glEnd()}
}
\end{document}

