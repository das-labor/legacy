% $Header: /cvsroot/latex-beamer/latex-beamer/solutions/generic-talks/generic-ornate-15min-45min.de.tex,v 1.4 2004/10/07 20:53:08 tantau Exp $

\documentclass{beamer}

\usepackage{graphicx}
\usepackage{amsfonts}
\usepackage{amsmath}
\newcommand{\It}[1]{\textit{#1}}
\newcommand{\Bf}[1]{\textbf{#1}}
\newcommand{\Tt}[1]{\texttt{#1}}
\newcommand{\mat}[1]{\mathbf{#1}}
\newcommand{\ten}[1]{\mathcal{#1}}
\newcommand{\set}[1]{\mathbb{#1}}
\renewcommand{\Vec}[1]{\overrightarrow{#1}}

% Diese Datei enth�lt eine L�sungsvorlage f�r:


% - Vortr�ge �ber ein beliebiges Thema.
% - Vortragsl�nge zwischen 15 und 45 Minuten. 
% - Aussehen des Vortrags ist verschn�rkelt/dekorativ.



% Copyright 2004 by Till Tantau <tantau@users.sourceforge.net>.
%
% In principle, this file can be redistributed and/or modified under
% the terms of the GNU Public License, version 2.
%
% However, this file is supposed to be a template to be modified
% for your own needs. For this reason, if you use this file as a
% template and not specifically distribute it as part of a another
% package/program, I grant the extra permission to freely copy and
% modify this file as you see fit and even to delete this copyright
% notice. 



\mode<presentation>
{
  \usetheme{Warsaw}
  % oder ...
  
  \setbeamercovered{transparent}
  % oder auch nicht
}


\usepackage[german]{babel}
% oder was auch immer

\usepackage[latin1]{inputenc}
% oder was auch immer

\usepackage{times}
\usepackage[T1]{fontenc}
% Oder was auch immer. Zu beachten ist, das Font und Encoding passen
% m�ssen. Falls T1 nicht funktioniert, kann man versuchen, die Zeile
% mit fontenc zu l�schen.


\title[]  % (optional, nur bei langen Titeln n�tig)
{Grundlagen der 3D-Grafik mit OpenGl}

\subtitle
{} % (optional)

\author[] % (optional, nur bei vielen Autoren)
{Dipl.-Inform.(FH) Martin Ongsiek}
% - Der \inst{?} Befehl sollte nur verwendet werden, wenn die Autoren
%   unterschiedlichen Instituten angeh�ren.

% - Der \inst{?} Befehl sollte nur verwendet werden, wenn die Autoren
%   unterschiedlichen Instituten angeh�ren.
% - Keep it simple, niemand interessiert sich f�r die genau Adresse.

\date[] % (optional)
{12.01.2006 - www.das-labor.org}


\subject{Informatik}
% Dies wird lediglich in den PDF Informationskatalog einf�gt. Kann gut
% weggelassen werden.


% Falls eine Logodatei namens "university-logo-filename.xxx" vorhanden
% ist, wobei xxx ein von latex bzw. pdflatex lesbares Graphikformat
% ist, so kann man wie folgt ein Logo einf�gen:

% \pgfdeclareimage[height=0.5cm]{university-logo}{university-logo-filename}
% \logo{\pgfuseimage{university-logo}}



% Folgendes sollte gel�scht werden, wenn man nicht am Anfang jedes
% Unterabschnitts die Gliederung nochmal sehen m�chte.
\AtBeginSubsection[]
{
  \begin{frame}<beamer>
    \frametitle{Gliederung}
    \tableofcontents[currentsection,currentsubsection]
  \end{frame}
}


% Falls Aufz�hlungen immer schrittweise gezeigt werden sollen, kann
% folgendes Kommando benutzt werden:

%\beamerdefaultoverlayspecification{<+->}



\begin{document}

\begin{frame}
  \titlepage
\end{frame}

\begin{frame}
  \frametitle{Gliederung}
  \tableofcontents
  % Die Option [pausesections] k�nnte n�tzlich sein.
\end{frame}



% Da dies ein Vorlage f�r beliebige Vortr�ge ist, lassen sich kaum
% allgemeine Regeln zur Strukturierung angeben. Da die Vorlage f�r
% einen Vortrag zwischen 15 und 45 Minuten gedacht ist, f�hrt man aber
% mit folgenden Regeln oft gut.  

% - Es sollte genau zwei oder drei Abschnitte geben (neben der
%   Zusammenfassung). 
% - *H�chstens* drei Unterabschnitte pro Abschnitt.
% - Pro Rahmen sollte man zwischen 30s und 2min reden. Es sollte also
%   15 bis 30 Rahmen geben.

\newcommand{\cx}{\cos(\alpha)}
\newcommand{\sx}{\sin(\alpha)}
\newcommand{\cy}{\cos(\beta)}
\newcommand{\sy}{\sin(\beta)}
\newcommand{\cz}{\cos(\gamma)}
\newcommand{\sz}{\sin(\gamma)}


\section{Einf�hrung}

\subsection{Ein wennig Mathematik}

\begin{frame}
\frametitle{Das OpenGl Koordinatensystem}
\begin{center}
        \includegraphics[width=10cm]{images/bild02.jpg}
\end{center}

\end{frame}

\begin{frame}
\frametitle{Homogene Koordinaten}
			\begin{equation}
				P = \left( \begin{array}{c}p_x\\p_y\\p_z\\p_w \end{array}\right)
			\end{equation}
\pause
\begin{itemize}
    \item $p_w$ ist �blicherweise $1$
    \pause
    \item Uniforme Behandlung von geometrischen Transformationen durch eine $4 \times 4$ Matrix
    \pause
    \item komplexe Transformationen k�nnen durch die Kombination von elementaren 
          Transformationen gebildet werden
\end{itemize}

\end{frame}


\begin{frame}
\frametitle{Translatieren}
            glTranslate($t_x$, $t_y$, $t_z$);
            \pause    
			\begin{equation}
				\mat{T}(t_x, t_y, t_z) ~=~ \left( \begin{array}{cccc}
				1&  0&  0&  t_x\\
				0&  1&  0&  t_y\\
				0&  0&  1&  t_z\\
				0&  0&  0&  1
				\end{array} \right)
			\end{equation} 
\end{frame}

\begin{frame}
\frametitle{Skalieren}
        glScale($s_x$, $s_y$, $s_z$);
        \pause    
        \begin{equation}
				\mat{S}(s_x, s_y, s_z) ~=~ \left( \begin{array}{cccc}
				s_x &0   &0   &0\\
				0   &s_y &0   &0\\
				0   &0   &s_z &0\\
				0   &0   &0   &1
				\end{array} \right)
        \end{equation} 
\end{frame}

\begin{frame}
\frametitle{Rotieren}
            glRotate(<Winkel in �>, <Anteil x>, <Anteil y>, <Anteil z>);
            \pause
			\begin{eqnarray}
			\mat{RX}(\alpha) &=& \left( \begin{array}{cccc}
				1    &0    &0    &0\\
				0    &\cx  &-\sx &0\\
				0	 &\sx  &\cx	 &0\\
				0	 &0	   &0	 &1
				\end{array} \right) \\
				\pause
			\mat{RY}(\beta) &=& \left( \begin{array}{cccc}
				\cy	 &0	   &\sy	 &0\\
				0	 &1	   &0	 &0\\
				-\sy &0	   &\cy	 &0\\
				0	 &0	   &0	 &1
				\end{array} \right)\\
				\pause
			\mat{RZ}(\gamma) &=& \left( \begin{array}{cccc}
				\cz	 &-\sz &0	 &0\\
				\sz	 &\cz  &0    &0\\
				0	 &0	   &1	 &0\\
				0	 &0	   &0	 &1
				\end{array} \right)
			\end{eqnarray}
\end{frame}

\subsection{Dreidimensional malen}

\begin{frame}
\frametitle{OpenGL Befehlssatz}

\begin{itemize}
	\item In Specifikation 1.1 ca. 150 recht simplen Funktionen
	\pause
	\item Arbeitet intern als State-Maschine
	\pause
	\item Alle Funktionen fangen mit gl an. Makros mit GL\_
	\pause
	\item Funktionen mit meheren Datentypen.
	\pause
	\item Suffix mit Anzahl und Art der Parameter. Z.B. glVertex3f(), glVertex4d()
\end{itemize}
\end{frame}


\begin{frame}
\frametitle{Prinzip}
Man schreibt Punktkoordinaten \\
glVertex*($x$, $y$, $z$);\\
\pause
glBegin(GL\_POINTS);\\ 
... hier hinein 	\\
glEnd(GL\_POINTS);\\ 		
\pause
um Punkte zu malen.
\end{frame}

\begin{frame}
\frametitle{Mehr als nur Punkte malen}

\begin{center}
        \includegraphics[width=12cm]{images/bild01.jpg}
\end{center}


\end{frame}

\begin{frame}
\frametitle{Oben und Unten?}
\pause
Punkte m�ssen gegen der Uhrzeigersinn, also in mathematische positiver Richtung definiert werden, damit OpenGl wei�, welche Seite oben ist und daher sichtbar sind.
\end{frame}

\begin{frame}
\frametitle{Problem bei Polygonen mit mehr als 3 Ecken}
\begin{itemize}
\item Alle Punkte m�ssen auf einer Ebene liegen.
\pause
\item Nur stumpfe Winkeln bei Polygonen mit mehr als 4 Ecken. 
\end{itemize}
\end{frame}

\begin{frame}
\frametitle{Einf�rben}
Mit glColor4f($c_R$, $c_G$, $c_B$, $c_A$); \\
setz man die aktuelle Farbe. 
\pause
\begin{itemize}
\item Farbwerte sind im Bereich $0 \le c \le 1$ definiert
\pause
\item Farbwert gilt bis zum erneuten setzen
\end{itemize}

\end{frame}

\subsection{Objekte miteinander verkn�pfen}

\begin{frame}
\frametitle{Pivotpunkt}
\begin{itemize}
\item Der Pivotpunkt dient als Ausgangspunkt f�r Transformationen. 
\end{itemize}

\end{frame}

\begin{frame}
\frametitle{Matrix Stack}

\begin{itemize}
\item glPushMatrix()
\pause
\item glPopMatrix();
\pause
\end{itemize}
Ben�tigt man, wenn an einem Objekt mehrere Teilobjekte h�ngen. 
\end{frame}

\section{Erweiterte Funktionen}

\begin{frame}
\frametitle{}
\end{frame}


\end{document}

