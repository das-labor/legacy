\section{Variablen}
\begin{frame}
	\frametitle{Beispiel 4}
    Ein einfaches Beispiel mit Variablen:
 %   \begin{block}
      \lstinputlisting{example4.make}
 %   \end{block}
 
  \begin{tabular}{|c|l|}
    \hline
    \textit{variable} = \textit{wert} & Zuweisung \\ \hline 
    \$(\textit{variable}) & Referenzierung \\ \hline
  \end{tabular}
\end{frame}

%\begin{frame}
%	\frametitle{Variablen definieren, anders}
%    \begin{lstlisting}
%define varname
%   ... text ...
%endef
%    \end{lstlisting}
%\end{frame}

\begin{frame}
	\frametitle{It's all about ...}
	\begin{itemize}
    \item<2-> \huge{Strings}
    \item<3-> \huge{Lists (of Strings)}
    \end{itemize}
\end{frame}

\begin{frame}
	\frametitle{Listen sind ...}
    \begin{Large}
	Listen sind:
    \begin{itemize}
    \item<2-> Strings, die Elemente sind durch Whitespace getrennt
    \end{itemize}
    \end{Large}
\end{frame}

%\begin{frame}
%	\frametitle{Listen sind ...}
%	Listen sind:
%    \item<2-> Strings, die Elemente sind durch Whitespace getrennt
%    \end{itemize}
%\end{frame}
