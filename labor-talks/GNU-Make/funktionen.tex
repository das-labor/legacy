\section{Funktionen}

% $(subst from,to,text )
% $(patsubst pattern,replacement,text )
% $(strip string )
% $(findstring find,in )
% $(filter pattern ...,text )
% $(filter-out pattern ...,text )
% $(sort list )
% $(word n,text )
% $(wordlist s,e,text )
% $(words text )
% $(firstword names ...)
% $(lastword names ...)

\begin{frame}
	\frametitle{subst}
	\begin{Large}\$(subst from,to,text)\end{Large}

    \bigskip
    Führt Stringersetzung durch.
 
    \bigskip 
    Beispiel: \\
	\$(subst, ee,EE,speed tree) $\longrightarrow$ 'spEEd trEE'
\end{frame}

\begin{frame}
	\frametitle{patsubst}
	\begin{Large}\$(patsubst pattern,replacement,text)\end{Large}

	\bigskip
    Führt Musterersetzung durch.

    \bigskip 
    Beispiel: \\
	\$(patsubst, \%.c,\%.d,code.c code.h edit.c) $\longrightarrow$ 'code.d code.h edit.d'
\end{frame}

\begin{frame}
	\frametitle{strip}
	\begin{Large}\$(strip string )\end{Large}

    \bigskip
	Entfernt Whitespace am Anfang und am Ende

    \bigskip 
    Beispiel: \\
	\$(strip,   code.c code.h edit.c  ) $\longrightarrow$ 'code.d code.h edit.d'
\end{frame}

\begin{frame}
	\frametitle{findstring}
	\begin{Large}\$(findstring find,in )\end{Large}

    \bigskip
	Sucht nach \textit{find} in \textit{in} und evaluiert zu \textit{find} wenn gefunden sonst zu ' ' (empty).

    \bigskip 
    Beispiel: \\
	\$(findsring, a, a b c) $\longrightarrow$ 'a'
\end{frame}

\begin{frame}
	\frametitle{filter}
	\begin{Large}\$(filter pattern ...,text )\end{Large}

    \bigskip
	Evaluiert zu einer Liste von Strings aus \textit{text} die alle auf eins 
   (oder mehere) der \textit{pattern} passen.

    \bigskip 
    Beispiel: \\
	\$(filter, \%.c \%.s, a.c b.h c.s) $\longrightarrow$ 'a.c c.s'
\end{frame}

\begin{frame}
	\frametitle{filter-out}
	\begin{Large}\$(filter-out pattern ...,text )\end{Large}

    \bigskip
	Evaluiert zu einer Liste von Strings aus \textit{text} die alle auf keins 
    der \textit{pattern} passen.

    \bigskip 
    Beispiel: \\
	\$(filter-out, \%.c \%.s, a.c b.h c.s) $\longrightarrow$ 'b.h'
\end{frame}

\begin{frame}
	\frametitle{sort}
	\begin{Large}\$(sort list )\end{Large}

    \bigskip
	Sortiert die Liste und entfernt doppelte Einträge.

    \bigskip 
    Beispiel: \\
	\$(sort, foo bar foo lose) $\longrightarrow$ 'bar foo lose'
\end{frame}

\begin{frame}
	\frametitle{word}
	\begin{Large}\$(word n, text )\end{Large}

    \bigskip
	Evaluiert zu dem n-ten String aus Text (Indizierung beginnend bei 1).

    \bigskip 
    Beispiel: \\
	\$(word 2, foo bar foo lose) $\longrightarrow$ 'bar'
\end{frame}

\begin{frame}
	\frametitle{words}
	\begin{Large}\$(words text )\end{Large}

    \bigskip
	Evaluiert zu der Anzahl von Strings in \textit{text}.

    \bigskip 
    Beispiel: \\
	\$(words foo bar foo lose) $\longrightarrow$ '4'
\end{frame}

\begin{frame}
	\frametitle{firstword}
	\begin{Large}\$(firstwordword names ... )\end{Large}

    \bigskip
	Evaluiert zu dem ersten String in \textit{names}.

    \bigskip 
    Beispiel: \\
	\$(firstword foo bar foo lose) $\longrightarrow$ 'foo'
\end{frame}

\begin{frame}
	\frametitle{lastword}
	\begin{Large}\$(lastwordword names ... )\end{Large}

    \bigskip
	Evaluiert zu dem letzten String in \textit{names}.

    \bigskip 
    Beispiel: \\
	\$(lastword foo bar foo lose) $\longrightarrow$ 'lose'
\end{frame}

\begin{frame}
	\frametitle{foreach}
	\begin{Large}\$(foreach name, list, expr )\end{Large}

    \bigskip
	Evaluiert für jeden String in \textit{list} \textit{expr} wobei der String über 
    die automatisch generierte Vaiable \textit{name} zugänglich ist. \textit{names}.

    \bigskip 
    Beispiel: \\
	\$(foreach var, bar foo lose, \$(var)\_x) $\longrightarrow$ 'bar\_x foo\_x lose\_x'
\end{frame}

\begin{frame}
	\frametitle{call}
	\begin{Large}\$(call function,parameters)\end{Large}

    \bigskip
	Ruft Funktion \textit{function} auf. Besonders sinvoll um nicht-standard
    Funktionen aufzurufen. (z.B. Funktionen aus der GMSL)

    \bigskip 
    Beispiel: \\
	\$(call uc,test) $\longrightarrow$ 'TEST'
\end{frame}

\begin{frame}
	\frametitle{eval}
	\begin{Large}\$(eval variable)\end{Large}

    \bigskip
	Evaluiert \textit{variable} im Makefile Kontext, d.h. \textit{variable} kann
    Makefile Konstrukte (z.B. Rule, Dependancys) beinhalten.

    \bigskip 
    Beispiel: \\
	\$(call uc,test) $\longrightarrow$ 'TEST'
\end{frame}
