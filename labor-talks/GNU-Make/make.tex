\section{Make}
\begin{frame}
	\frametitle{Aufruf}
    \begin{itemize}
      \item<2->[] \texttt{make [..opts..] [target] \{target\}} 
    \end{itemize}
    Normalerweise wird dem Make-Aufruf die Liste der Ziele (targets) übergeben, dabei kann es sich handeln um:
    \begin{itemize}
      \item<3-> \underline{Dateien} die zu erstellen sind, oder
      \item<4-> spezielle \underline{Targets} die Instruktionen enthalten (Bsp. \textit{clean})
    \end{itemize}
    Wenn kein Target angegeben wird ist der Default \textit{all}.
    Die Dokumentation der Optionen findest du in der Manpage.
\end{frame}

\begin{frame}
	\frametitle{Beispiel 1}
    Ein einfaches Beispiel:
 %   \begin{block}
      \lstinputlisting{example1.make}
 %   \end{block}
\end{frame}

\begin{frame}
	\frametitle{Beispiel 2}
    Ein einfaches Beispiel mit automatischen Variablen:
 %   \begin{block}
      \lstinputlisting{example2.make}
 %   \end{block}
\begin{tabular}{|c|l|}
\hline
 Variable & Wert \\ \hline \hline
 \$@ & target \\ \hline 
 \$\^ & Abhängigkeiten \\ \hline
 \$$<$ & erste Abhängigkeit \\ \hline

\end{tabular}
\end{frame}

\begin{frame}
	\frametitle{Beispiel 3}
    Ein einfaches Beispiel mit generischen Regeln:
 %   \begin{block}
      \lstinputlisting{example3.make}
 %   \end{block}
\end{frame}