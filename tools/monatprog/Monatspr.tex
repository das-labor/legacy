% Labor Monatsprogramm
%
% LaTeX Code under GPL
% tw=80
%
% Todo:
%
%
%

\documentclass[10pt,landscape,a4paper]{article}
\usepackage{multicol}
\usepackage{calc}
\usepackage{graphicx}


% Don't print section numbers
\setcounter{secnumdepth}{0}

% All this page size stuff is a hack, but it seems to work
% Turn off header and footer
\pagestyle{empty}

\setlength{\oddsidemargin}{0.25in}
\setlength{\evensidemargin}{0.5in}
\setlength{\textwidth}{10in}


\setlength{\topmargin}{-0.75in}
\setlength{\textheight}{7.25in}
\setlength{\headheight}{0in}
\setlength{\headsep}{0in}




% termin options: time title text
\def\termin#1#2#3{\vskip 0.2cm {\it#1\dotfill#2\/}\\#3\par}



\begin{document}
\begin{multicols}{3}

%\hrule\vskip 3pt
\title{Termine im September 2006}

\section{Montags}
\termin{Bochumer GNU/Linux User Group}{04.09.2006 --- 19:00}{}
\termin{Bochumer GNU/Linux User Group}{18.09.2006 --- 19:00}{Zweites Treffen}

\section{Dienstags}
\termin{CCC Ruhrpott}{05.09.2006 --- 19:00}{}
\termin{OS Designs: GNU Hurd}{12.09.2006 --- 19:30}{}
\termin{L4 Microkernel Design}{26.09.2006 --- 19:30}{}

\section{Mittwochs}
\termin{LABOR Open Meeting}{06.09.2006 --- 19:30}{}
\termin{LABOR Bootstrap Meeting}{13.09.2006 --- 19:30}{}
\termin{LABOR Open Meeting}{20.09.2006 --- 19:30}{}
\termin{LABOR Open Meeting}{27.09.2006 --- 19:30}{}

\section{Donnerstags}
\termin{VHDL und FPGAs Teil 2}{07.09.2006 --- 19:30}{}
\termin{SOCCA 1}{06.09.2006 --- 19:30}{}
\termin{FUD --- The Movie}{06.09.2006 --- 19:30}{}
\termin{SOCCA 2}{06.09.2006 --- 19:30}{}


% ABOUT US
\section{\"Uber uns}

{\bf Konsumgewohnheiten vs. Rabattpunkte}\\
Alle Menschen verschenken ihre Privatsph\"are f\"ur ein paar
Merchandising-Artikel? Keiner versteht, dass Du nicht Deine
Konsumgewohnheiten f\"ur ein paar Rabattpunkte offenlegen m\"ochtest? Keiner
denkt dar\"uber nach, was man mit einer zentralen Fingerabdruckdatenbank
aller EU-B\"urger alles falsch machen kann? Keinen interessiert es, dass
jeder Informationsseitenabruf und -kontakt bald jahrelang gespeichert
wird? Denkst DU! Wir sollten uns dar\"uber unterhalten!\\
Dar\"uber, und auch \"uber Fragen wie "Kann das Konzept der
‘Kulturflatrate’ \"uberhaupt funktionieren oder stirbt die kulturelle
Vielfalt dann gleich mit?”, “Was bringen RFID- Erfassungsger\"ate an
Fußg\"angerampeln?”, "Wie k\"onnen offene B\"urgernetze als Alternative
zum Internet gestaltet werden?" oder auch “Kann man mit einem Trusted
Platform Module auch was Sinnvolles anfangen?“\\
\\
% grafik: eip

{\bf Wer bastelt hat Recht}\\
Das LABOR ist ein Ort, an dem in erster Linie gemacht und gedacht wird:
Wir benutzen und entwickeln freie Software; wir l\"oten, \"atzen und
programmieren Mikrocontrollerschaltungen; basteln Antennen; denken uns
praktikable L\"osungen f\"ur einen gesellschaftlichen Umgang mit vorhandener
oder sich entwickelnder Technik aus - wir haben den Anspruch mit Technologie
Neues und Sinnvolles zu gestalten.\\
Das LABOR ist dynamisch, seine Strukturen nicht fest. Was in und mit
ihm passiert, h\"angt auch von Dir ab. Du willst etwas ver\"andern oder
verbessern? Technik ausprobieren oder \"uber deren Einsatzm\"oglichkeiten
lernen? - Oder einfach nur Leute kennenlernen, die das auch tun? - Dann
komm' vorbei und mach mit - das LABOR entwickelt sich mit Dir!\\
{\bf Lerne die Regeln, damit du weißt, wie man sie bricht}\\
Wichtiger als Hardware und Equipment sind Menschen, die wissen, wie das alles
funktioniert. Im Labor gibt es Vortr\"age, Workshops und Diskussionen zu
den unterschiedlichsten Technologien. Wenn keine Veranstaltung stattfindet,
bastelt man - zusammen oder alleine. Aber immer tauscht man sein Wissen: Denn
alles, was Dir zeigt, wie die Welt funktioniert, hat hier seinen Platz.\\
\\
{\bf N\"achster Termin f\"ur Hereingucker}\\
Komm doch einfach zu einem unserer Open Meetings vorbei! Am besten n\"achsten
Mittwoch abends so ab 19.30 Uhr.\\
\\
\section{freifunk}
Der Bochumer Freifunk ist eine nicht-kommerzielle, für jeden
offene Initiative zur Installation eines öffentlich zugänglichen
W-Lan-Netzes in Bochum sowie am Campus und den Studierendenwohnheimen
der Ruhr-Universität. Die Vision der Freifunk-Communtiy FREIFUNK.net ist
die Demokratisierung der Kommunikationsmedien und die Förderung lokaler
Sozialstrukturen durch freie InternetNetzwerke durch die Luft mittels
Wireless Lan. Zum Aufbau dieses Netzes treffen wir uns regelmäßig
im Bochumer Labor. Du lernst mit vielen anderen wie man Antennen baut,
Wirless am Laptop und PC einsetzt, Router einrichtet, Bionade, Mate und
Afrikola trinkt oder trägt sowie sich am Bochumer Funknetz beteiligen
kann mit seinem Router. Immer Montags im Labor.\\



% AUFMACHER

% grafik: logo
\section{Programm September 2006} % dyna
Jetzt! Schnell! Terminkalender aufschlagen! In der Hand h\"altst du den
Veranstaltungskalender des LABORs. Du solltest besser mal reinschauen,
Dir einen Stift schnappen und Dir vormerken, wann DU vorbeischaust! \\
\\
Das LABOR ist Dein Raum in Bochums Innenstadt, in dem Platz ist f\"ur
Dinge, die Du zu Hause nicht tun kannst. Hier triffst Du andere Leute,
die mit Technik kreativ, konstruktiv und kritisch umgehen. Hier ist
Deine Infrastruktur, Dein WLAN, Dein L\"otkolben, Deine Bastelecke. Du
kannst Vortr\"age h\"oren, an Workshops teilnehmen, oder selber welche
veranstalten. Join us! \\

% grafik: kabels

% RECHTLICHES
Monats-Programm LABOR, Ausgabe Nr. 2006-09 \\ % dyna
Herausgeber: LABOR e.V., Rottstr. 31, 44793 Bochum \\
ViSdP/Chefredaktion: Felix Gr\"obert.\\
http://das-labor.org/

% grafik: anfahrt


\end{multicols}
\end{document}
