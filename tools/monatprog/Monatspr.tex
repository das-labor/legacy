% Labor Monatsprogramm
%
% thanks to Laurent Gr\'egoire VIM Refcard
%
% code under gpl
%
% Command for PDF : pdftex vimqrc.tex

%\documentclass[a4paper]{article}

\pdfoutput=1
\pdfpageheight=21cm
\pdfpagewidth=29.7cm


% Font definitions
\def\scaledmag#1{ scaled \magstep #1}

\font\bigbf=cmb12
\font\smallrm=cmr8
\font\smalltt=cmtt8
\font\tinyit=cmmi5
\font\headingfont=cmbx10 \scaledmag1


\def\title#1{\hfil{\bf #1}\hfil\par\vskip 2pt\hrule}
\def\cm#1#2{{\tt#1}\dotfill#2\par}
\def\cn#1{\hfill$\lfloor$ #1\par}
\def\sect#1{\vskip 0.7cm {\headingfont #1}\mark{#1}\par}
\def\para#1{\vskip 0.2cm {#1\/}\par}

% termin options: time title text
\def\termin#1#2#3{\vskip 0.2cm {\it#1\dotfill#2\/}\par#3\par}

% Three columns definitions
\parindent 0pt
\nopagenumbers
\hoffset=-1.56cm
\voffset=-1.54cm
\newdimen\fullhsize
\fullhsize=27.9cm
\hsize=8.5cm
\vsize=19cm
\def\fullline{\hbox to\fullhsize}
\let\lr=L
\newbox\leftcolumn
\newbox\midcolumn
\output={
  \if L\lr
    \global\setbox\leftcolumn=\columnbox
    \global\let\lr=M
  \else\if M\lr
    \global\setbox\midcolumn=\columnbox
    \global\let\lr=R
  \else
    \tripleformat
    \global\let\lr=L
  \fi\fi
  \ifnum\outputpenalty>-20000
  \else
    \dosupereject
  \fi}
\def\tripleformat{
  \shipout\vbox{\fullline{\box\leftcolumn\hfil\box\midcolumn\hfil\columnbox}}
  \advancepageno}
\def\columnbox{\leftline{\pagebody}}


% Card content
% Header
%\hrule\vskip 3pt
\title{Termine im September 2006}

\sect{Montags}
\termin{Bochumer GNU/Linux User Group}{04.09.2006 --- 19:00}{}
\termin{Bochumer GNU/Linux User Group}{18.09.2006 --- 19:00}{Zweites Treffen}

\sect{Dienstags}
\termin{CCC Ruhrpott}{05.09.2006 --- 19:00}{}
\termin{OS Designs: GNU Hurd}{12.09.2006 --- 19:30}{}
\termin{L4 Microkernel Design}{26.09.2006 --- 19:30}{}

\sect{Mittwochs}
\termin{LABOR Open Meeting}{06.09.2006 --- 19:30}{}
\termin{LABOR Bootstrap Meeting}{13.09.2006 --- 19:30}{}
\termin{LABOR Open Meeting}{20.09.2006 --- 19:30}{}
\termin{LABOR Open Meeting}{27.09.2006 --- 19:30}{}

\sect{Donnerstags}
\termin{VHDL und FPGAs Teil 2}{07.09.2006 --- 19:30}{}
\termin{SOCCA 1}{06.09.2006 --- 19:30}{}
\termin{FUD --- The Movie}{06.09.2006 --- 19:30}{}
\termin{SOCCA 2}{06.09.2006 --- 19:30}{}


\sect{\"Uber uns}

{\bf Konsumgewohnheiten vs. Rabattpunkte}\par
Alle Menschen verschenken ihre Privatsph\"are f\"ur ein paar Merchandising-Artikel? Keiner versteht, dass Du nicht Deine Konsumgewohnheiten f\"ur ein paar Rabattpunkte offenlegen m\"ochtest? Keiner denkt dar\"uber nach, was man mit einer zentralen Fingerabdruckdatenbank aller EU-B\"urger alles falsch machen kann? Keinen interessiert es, dass jeder Informationsseitenabruf und -kontakt bald jahrelang gespeichert wird? Denkst DU! Wir sollten uns dar\"uber unterhalten!\par
Dar\"uber, und auch \"uber Fragen wie "Kann das Konzept der ‘Kulturflatrate’ \"uberhaupt funktionieren oder stirbt die kulturelle Vielfalt dann gleich mit?”, “Was bringen RFID- Erfassungsger\"ate an Fußg\"angerampeln?”, "Wie k\"onnen offene B\"urgernetze als Alternative zum Internet gestaltet werden?" oder auch “Kann man mit einem Trusted Platform Module auch was Sinnvolles anfangen?“\par
\vskip1cm

% XXX \psfig{file=fig1.eps,width=\textwidth}

{\bf Wer bastelt hat Recht}\par
Das LABOR ist ein Ort, an dem in erster Linie gemacht und gedacht wird: Wir benutzen und entwickeln freie Software; wir l\"oten, \"atzen und programmieren Mikrocontrollerschaltungen; basteln Antennen; denken uns praktikable L\"osungen f\"ur einen gesellschaftlichen Umgang mit vorhandener oder sich entwickelnder Technik aus - wir haben den Anspruch mit Technologie Neues und Sinnvolles zu gestalten.\par
Das LABOR ist dynamisch, seine Strukturen nicht fest. Was in und mit ihm passiert, h\"angt auch von Dir ab. Du willst etwas ver\"andern oder verbessern? Technik ausprobieren oder \"uber deren Einsatzm\"oglichkeiten lernen? - Oder einfach nur Leute kennenlernen, die das auch tun? - Dann komm' vorbei und mach mit - das LABOR entwickelt sich mit Dir!\par
{\bf Lerne die Regeln, damit du weißt, wie man sie bricht}\par
Wichtiger als Hardware und Equipment sind Menschen, die wissen, wie das alles funktioniert. Im Labor gibt es Vortr\"age, Workshops und Diskussionen zu den unterschiedlichsten Technologien. Wenn keine Veranstaltung stattfindet, bastelt man - zusammen oder alleine. Aber immer tauscht man sein Wissen: Denn alles, was Dir zeigt, wie die Welt funktioniert, hat hier seinen Platz.\par
\vskip1cm
{\bf N\"achster Termin f\"ur Hereingucker}\par
Komm doch einfach zu einem unserer Open Meetings vorbei! Am besten n\"achsten Mittwoch abends so ab 19.30 Uhr.\par
\vskip1cm

Programm Juli 2006
Jetzt! Schnell! Terminkalender aufschlagen! In der Hand h\"altst du den Veranstaltungskalender des LABORs. Du solltest besser mal reinschauen, Dir einen Stift schnappen und Dir vormerken, wann DU vorbeischaust!
Das LABOR ist Dein Raum in Bochums Innenstadt, in dem Platz ist f\"ur Dinge, die Du zu Hause nicht tun kannst. Hier triffst Du andere Leute, die mit Technik kreativ, konstruktiv und kritisch umgehen. Hier ist Deine Infrastruktur, Dein WLAN, Dein L\"otkolben, Deine Bastelecke. Du kannst Vortr\"age h\"oren, an Workshops teilnehmen, oder selber welche veranstalten. Join us!

% footer
\vfill \hrule\smallskip
{\smallrm
Monats-Programm LABOR, Ausgabe Nr. {\oldstyle 2006-09} \par % dyna
Herausgeber: LABOR e.V., Rottstr. 31, 44793 Bochum \par
ViSdP/Chefredaktion: Felix Gr\"obert.\par
{\smalltt http://das-labor.org/}}

% Ending
\supereject
\if L\lr \else\null\vfill\eject\fi
\if L\lr \else\null\vfill\eject\fi
\bye

% EOF

