\documentclass[11pt,a4paper,smallheadings,twocolumn,headexclude,footexclude]{scrartcl}
%\documentclass[11pt,a4paper,smallheadings,headexclude,footexclude]{scrartcl}
%\documentclass[10pt,a4paper,smallheadings,headexclude,footexclude]{scrartcl}
%\usepackage[utf8]{inputenc}
\usepackage{a4wide}
%\usepackage{ucs}
\usepackage{longtable}
\usepackage{amssymb,amsmath}
\usepackage{amsfonts}
\usepackage{amssymb}
\usepackage[latin1]{inputenc}
\usepackage{ngerman}
\usepackage{graphicx}

% Hiermit werden wichtige Thesen gekennzeichnet
\newcommand{\these}[1]{\begin{quotation}\vspace{0.5cm}\framebox{\parbox{135mm}{\begin{center}#1\end{center}}}\marginpar{\begin{Huge}\textbf{!}\end{Huge}}\vspace{0.5cm}\end{quotation}}


% Hier kann man Abbildungen einbetten
\newenvironment{abbildung}[1]{\begin{figure}[h]  \caption{#1}\hline \vspace{0.25cm} \begin{center} }{\end{center} \vspace{0.25cm} \hline \end{figure} }

\begin{document}
\begin{center}
\begin{huge}
MICROCONTROLLER PROGRAMMIERUNG
\end{huge}\\\vspace{2mm}
\large{- Eine kurze Anleitung -}
\end{center}
\vspace{2mm}
\section*{Bit Arithmetik}
{\bf Logisches UND ''\&'':}
	Das logische \& setzt nur Bits, die rechts- UND linksseitig logisch \texttt{1}
	sind.
	\\\texttt{0101 \& 0011 $\rightarrow$ 0001}

\vspace{2mm}

{\bf Logisches ODER ''$|$'':}
Das logische Oder setzt alle Bits die links- und/oder rechtsseitig gesetzt sind.
\\\texttt{0101 $|$ 0011 $\rightarrow$ 0111}

\vspace{2mm}

{\bf Invertierung ''$\sim$'':}
Die Tilde ($\sim$) negiert das folgende Argument.
\\\texttt{$\sim$0101 $\rightarrow$ 1010}

\vspace{2mm}

{\bf Das XOR ''\textasciicircum'':}
Die XOR Funktion setzt alle Bits, die nur links- ODER rechtsseitig
gesetzt sind. Beidseitig gleich gesetzte Bits werden zu \texttt{0}.
\\\texttt{0101{\textasciicircum}0011 $\rightarrow$ 0110}

\vspace{2mm}

{\bf Der Shift-Operator ''$>>$'' und ''$<<$'':}
Mit dem Shift Operator lassen sich Bits nach Links oder Rechts verschieben. Das Argument rechtsseitig des
Operators gibt dabei an, wie oft geschoben werden soll. "Uberstehende Bits fallen dabei weg und neue Bits
werden auf \texttt{0} gesetzt.
\\\texttt{0101$<<$1 $\rightarrow$ 1010}
\\\texttt{0101$>>$2 $\rightarrow$ 0001}

\subsection*{Operatoren und Variablen}
Es ist auch m"oglich, alle Operatoren auf eine Variable oder Port anzuwenden. Die
Instruktion ''\texttt{PORTA$>>=$2}'' verschiebt alle Bits des Port A um 2 Stellen nach rechts.
\\
Die Instruktion ''\texttt{PORTA$|=$4}'' setzt das 3. Bit von rechts auf \texttt{1}.

\subsection*{Das Byte}
Ein Byte besteht aus 8 Bits:\\
\begin{centering}
\begin{tabular}{|c|c|c|c|c|c|c|c|}
\hline
$2^7$ &
$2^6$ &
$2^5$ &
$2^4$ &
$2^3$ &
$2^2$ &
$2^1$ &
$2^0$ \\
\hline
\end{tabular}
\end{centering}

\section*{Hexadezimalschreibweise}
In C ist es nicht m"oglich, direkt Bit-Variablen zu setzen. Daher ist es n"otig, diese entweder
im Dezimalsystem oder Hexadezimalsystem anzugeben. Das Hexadezimalsystem eignet sich besonders
gut, da man relativ einfach umrechnen kann.\\
\begin{tabular}{c c c | c c c}
Dez. & Hex & Bit & Dez. & Hex & Bit \\
\hline
0 & 0 & 0000 & 8 & 8 & 1000 \\
1 & 1 & 0001 & 9 & 9 & 1001 \\
2 & 2 & 0010 &10 & A & 1010 \\
3 & 3 & 0011 &11 & B & 1011 \\
4 & 4 & 0100 &12 & C & 1100 \\
5 & 5 & 0101 &13 & D & 1101 \\
6 & 6 & 0110 &14 & E & 1110 \\
7 & 7 & 0111 &15 & F & 1111 \\
\end{tabular}
\subsection*{Beispiele}
\begin{tabular}{c | c}
\texttt{0xA5 $\leftrightarrow$ 1010 0101} &
\texttt{0x17 $\leftrightarrow$ 0001 0111} \\

\texttt{0x00 $\leftrightarrow$ 0000 0000} &
\texttt{0xFF $\leftrightarrow$ 1111 1111} \\

\texttt{0x0F $\leftrightarrow$ 0000 1111} &
\texttt{0x64 $\leftrightarrow$ 0110 0100} \\
\end{tabular}
\section*{Ports \& Pins}
Als ein \emph{Port} werden immer 8 (oder weniger) Pinne zusammengefasst. Bei AVR Controllern
sind diese ''\texttt{PORTA, PORTB}'' usw. benannt.
\begin{itemize}
	\item Pin A1 auf Ausgang schalten:\\\texttt{DDRA |= \_BV(PA1);}
	\item Pullup an Pin D0 anschalten: \\\texttt{PORTD |= \_BV(PD0)}
		\\\emph{Wichtig: Der Pullup kann erst eingeschaltet werden, NACHDEM
		der Pin auf Eingang geschaltet wurde}
	\item Pin A4 auslesen: \\\texttt{if (PINA \& \_BV(PA4)) \{ ... }
	\item Pin A3 auf high schalten:\\\texttt{PORTA |= \_BV(PA3)}
	\item Pin A3 auf low schalten:\\\texttt{PORTA \&= $\sim$(\_BV(PA3))}
	\item Pin A3 umschalten (toggle):\\\texttt{PORTA {\textasciicircum}= \_BV(PA3)}
\end{itemize}

\section*{Interrupts}
Interrupts unterbrechen den Fluss des Hauptprogramms. Sie k"onnen durch verschiedenste
Quellen ausgel"ost werden (z.B. Timer oder Signal am Interrupt Pin). Um einen Interrupt
zu benutzen muss man in der Regel folgende Dinge tun:
\begin{enumerate}
	\item Interrupt Routine definieren und mit Code f"ullen:
		\\\texttt{ISR(...) \{ ... \}}
	\item Entsprechende Bits in entsprechenden Registern setzen ($\rightarrow$ siehe Datenblatt).
		z.B. f"ur den ext. Interrupt 0:\\
		\texttt{MCUCR |= \_BV(ISC01);\\GICR |= \_BV(INT0);}
		\\
		Meist muss man hier 2 oder 3 Register setzen.
	\item Interrupts anschalten: \texttt{sei();}
\end{enumerate}

\section*{Toolchain benutzen}
\begin{enumerate}
	\item Einstellungen f"ur Deinen Programmer setzen
		$\rightarrow$ \texttt{Makefile} editieren
	\item Compilieren mit dem Befehl {\bf\texttt{make}}.
	\item Controller flashen (entweder mit avrdude oder \texttt{make flash} (je nach Makefile))
\end{enumerate}

\section*{Fuse Bits}
Fuse bits setzen die wichtigsten Einstellung des Controllers - z.B. mit welchem Takt
er laufen soll und aus welcher Quelle er seinen Takt beziehen soll (intern, externer
Quartz, ...).\\
Wie die Fuse bits zu setzen sind, ist dem Datenblatt zu entnehmen. Vorsicht:
\emph{Falsche Einstellungen k"onnen den Controller unbrauchbar machen}.
\subsection*{Fuse Bits mit avrdude setzen}
\texttt{avrdude -p m32 -c usbasp $\backslash$\\-U lfuse:w:0x9F:m -U hfuse:w:0xC9:m}


\section*{Checkliste mit den h"auftigsten Fehlern}
\begin{itemize}
	\item \emph{Nutzereingaben funktionieren nicht wie gew"unscht.}
		$\rightarrow$ Taster entprellt? Pullup an? Liest Du auch wirklich
		das \texttt{PIN} Register (und nicht \texttt{PORT}) aus?
	\item \emph{Programm l"auft langsam / Ausg"ange verhalten sich komisch.}
		$\rightarrow$ Fuses richtig gesetzt?
	\item \emph{Der ADC verh"alt sich komisch / misst nicht richtig / ''zappelt''.}
		$\rightarrow$ Kondensator an AVCC geschaltet?
	\item \emph{Der Timer funktioniert nicht.}
		$\rightarrow$ Interrupts eingeschaltet (\texttt{sei()})? Bit im \texttt{TIMSK} Register gesetzt?
\end{itemize}

\section*{Wichtige Links}
{\bf Laborboard Dokumentation}:\\\texttt{https://das-labor.org/wiki/Laborboard}\\
\\
{\bf Laborboard Beispielcode}:\\
\texttt{https://www.das-labor.org/trac/browser/\\microcontroller/src-atmel/tests/helloboard}


\end{document}
